\chapter{OTV}
\section{Contesto di utilizzo}

Come già brevemente descritto, OTV prevede un deployment su dispositivi di dimensioni ridotte e autosufficienti dal punto di
vista energetico. In particolare, la configurazione testata su Raspberry introduceva i seguenti componenti:
\begin{itemize}
    \item Raspberry Pi 3B/4 per l'elaborazione
    \item Panello solare 6 W (PiJuice Solar Panel) per sostenere i consumi energetici
    \item Batteria esterna (PiJuice Hat) per fornire alimentazione
\end{itemize}
\cite{app11157027} Una simile configurazione verrebbe usata con smartphone, salvo ovviamente l'utilizzo di una batteria 
aggiuntiva.

\underline{\emph{*Immagine applicazione OTV 2 (quella con lo sketch del fiume usata nel paper nuovo)*}}

\section{Ciclo di funzionamento}
Il dispositivo così composto, una volta accuratamente posizionato ed avviato, dovrebbe eseguire \emph{quattro} misurazioni 
della velocità dell'acqua ogni ora, risultando quindi a regime in un ciclo di funzionamento periodico della durata di 15 minuti.

Sebbene la misurazione mediante l'algoritmo OTV sia svolta sul momento, non viene effettuata sulle immagini direttamente ricevute
e lette in input dalla telecamera: il video acquisito necessita di una fase preliminare che prepari le immagini per essere elaborate.
Questo viene fatto, tra le altre cose, per consentire di scegliere un settaggio particolare (ad esempio, selezionare una 
risoluzione diversa rispetto al video originale), utile successivamente al fine di ottimizzare l'elaborazione.

Il ciclo di funzionamento si articola quindi in questo modo:
\begin{enumerate}
    \item Fase di \textbf{acquisizione}: le immagini vengono acquisite dalla telecamera. Questa fase ha una durata fissa e dipende
    dalla lunghezza del video che si vuole analizzare: tipicamente 20 secondi.
    \item Fase di \textbf{estrazione} dei frame: a partire dal video acquisito, si estraggono i fotogrammi che lo compongono a seconda
    della configurazione scelta, in particolare è possibile specificare la risoluzione desiderata tra:
    \begin{itemize}
        \item Full Resolution (\textbf{F}): Risoluzione originale
        \item Half Resolution (\textbf{H}): Risoluzione dimezzata
        \item Quarter Resolution (\textbf{Q}): Risoluzione 1/4 dell'originale
    \end{itemize}
    \item Fase di \textbf{elaborazione} (OTV): a questo punto le immagini estratte vengono effettivamente elaborate utilizzando
    OTV. Questa fase è cruciale dal punto di vista dei consumi in quanto è quella che può variare maggiormente a seconda della
    configurazione usata e delle ottimizzazioni implementate. È bene quindi analizzarla di conseguenza.
    \item Fase di \textbf{idle}: una volta conclusa l'elaborazione (ed eventualmente spediti i dati rilevati) segue un periodo di stand-by,
    in cui si attende il tempo necessario prima della prossima rilevazione. Anche questa fase è molto importante per determinare
    i consumi energetici del processo: se il dispositivo dovesse disporre di una modalità di risparmio energetico, 
    l'energia utilizzata potrebbe diminuire drasticamente.
\end{enumerate}

Le fasi su cui è possibile effettivamente lavorare per ottenere risultati migliori sono quelle di elaborazione (in modo particolare)
e di idle.

\section{Ottimizzazioni}

Le ottimizzazioni applicabili ad OTV analizzate in \cite{rs12122047} consistono in una serie di tecniche e meccanismi che
possano contribuire ad aumentare l'efficienza energetica del sistema. %e in generale ad abbassarne i consumi. 
Tra queste, possiamo distinguere quelle legate al \textbf{software} e quelle invece a livello \textbf{hardware} 
(ad esempio, l'utilizzo di istruzioni particolari).

Le ottimizzazioni software consistono essenzialmente nella configurazione del dispositivo in modo ad esempio
da disattivare le opzioni software che risultino superflue (Wi-Fi, Bluetooth ecc.). Si parla di ottimizzazioni derivanti dal
sistema operativo utilizzato e dunque dipendenti dal dispositivo in questione. Si vedranno ottimizzazioni di questo tipo
esclusive al sistema Android.

\subsection{Ottimizzazioni hardware}

Nel caso delle ottimizzazioni hardware, si parla di particolari metodi che introducono differenti modelli di esecuzione
a livello di processore, facendo leva specialmente sulla parallelizzazione delle istruzioni. Questo, oltre agli ovvi vantaggi
in termini di performance, può portare ad una maggiore efficienza in termini di consumi.
Si delineano tre possibilità principali, eventualmente sovrapponibili, focalizzate su aspetti e modalità diverse di 
parallelizzazione:
\begin{itemize}
    \item Esecuzione multi-core mediante \textbf{OpenMP} o \textbf{TBB}: viene sfruttata l'architettura a più core per
    ottenere il più tipico livello di parallelismo. 
    \item Esecuzione di istruzioni \textbf{SIMD} tramite \textbf{NEON}
    \item Esecuzione su \textbf{GPU} mediante la libreria \textbf{OpenCL}
\end{itemize}

\subsubsection{OpenMP e TBB}




\clearpage{\pagestyle{empty}\cleardoublepage}