\chapter{Processo di sviluppo}

Parlando ora dell'effettiva operazione di porting di OTV su dispositivo Android, si espongono gli strumenti utilizzati
e le complessità riscontrate durante il processo di sviluppo. Verranno trattati inoltre i dettagli tecnici dell'applicazione
e i meccanismi utilizzati propri del sistema Android.

\section{OpenCV}

Come brevemente spiegato nel capitolo precedente, il codice disponibile di OTV è basato su OpenCV.
Nel caso di Android, i moduli di OpenCV devono essere necessariamente incapsulati all'interno dell'applicazione, non essendo
possibile installarli tra le librerie di sistema. Chiaramente, i moduli dovranno essere compilati con la toolchain di Android
per poter ottenere binari effettivamente compatibili con il sistema operativo.

I binari di OpenCV sono reperibili sul sito ufficiale oppure compilabili manualmente tramite una
\textbf{cross-compilazione}: questa seconda opzione è preferibile in quanto consente di configurare a proprio piacimento
l'installazione di OpenCV, aggiungendo o rimuovendo selettivamente le ottimizzazioni introdotte in \autoref{sec:ottim}.

Tutte le ottimizzazioni hardware discusse, infatti, sono disponibili su OpenCV. 
Non avrebbe senso invece implementarle nel codice OTV poiché tutte le operazioni che facciano uso intensivo della CPU e 
che richiedano tempi di esecuzione piuttosto lunghi sono interne alla libreria OpenCV.
L'unica ottimizzazione che effettivamente viene implementata nel codice di OTV è l'elaborazione delle immagini in scala di
grigi (non essendo a livello hardware): è sufficiente aggiungere una riga di codice nel momento in cui i frame estratti vengono
letti dall'applicazione.

\subsection{Cross-compilazione}

Dovendo testare numerose configurazioni diverse, per la cross-compilazione di OpenCV è stato preparato un semplice script 
in bash che automatizzasse in minima parte il processo e fornisse inoltre uno scheletro da seguire nel caso in cui si 
volesse modificare la configurazione di OpenCV, in particolare aggiungendo o rimuovendo ottimizzazioni hardware. 
Si riporta di seguito il passaggio principale dello script:

\definecolor{bg}{rgb}{0.9,0.9,0.9}

\usemintedstyle{tango}
\begin{minted}[
    frame=lines,
    framesep=2mm,
    baselinestretch=1.2,
    bgcolor=bg,
    fontsize=\footnotesize
]{bash}
    for ABI in "armeabi-v7a" "arm64-v8a"; do
        mkdir -p build_$ABI
        cd build_$ABI
        rm CMakeCache.txt
        cmake \
        -DCMAKE_TOOLCHAIN_FILE=$ANDROID_TOOLCHAIN_PATH \
        -DANDROID_ABI=$ABI \
        -DANDROID_NATIVE_API_LEVEL=$ANDROID_API_LEVEL \
        -DANDROID_STL=c++_shared \
        -GNinja -DCMAKE_BUILD_TYPE:STRING=Release \
        -DBUILD_FAT_JAVA_LIB=OFF \
        -DBUILD_JAVA=OFF \
        -DBUILD_SHARED_LIBS=ON \
        -DWITH_TBB=ON -DBUILD_TBB=ON \
        -DENABLE_NEON=OFF \
        -DWITH_OPENMP=OFF \
        -DWITH_OPENCL=OFF \
        -DWITH_CAROTENE=OFF \
        -DWITH_PTHREADS_PF=OFF \
        -DANDROID_SDK_ROOT=$ANDROID_SDK_ROOT \
        -DANDROID_NDK=$ANDROID_NDK \
        -DOPENCV_ENABLE_NONFREE=ON \
        -DBUILD_TESTS=FALSE -DBUILD_PERF_TESTS=FALSE \
        -DBUILD_ANDROID_EXAMPLES=FALSE \
        -DBUILD_DOCS:BOOL=OFF -DWITH_IPP=OFF -DWITH_MSMF=OFF \
        ..
    
        ninja
        ninja install
        cd ..
    done
\end{minted}

In questo caso viene utilizzato \textbf{CMake} per configurare la compilazione (scelta obbligata, poiché il file CMake è fornito 
nella distribuzione di OpenCV) e \textbf{ninja} come \textit{build system}. 
CMake infatti non esegue la vera e propria compilazione, ma prepara l'ambiente e crea i file di compilazione che 
verranno poi usati dal build system per effettivamente compilare il progetto. In alternativa a ninja era possibile 
utilizzare \textbf{Make}, la scelta fra i due non comporta nessuna differenza nel risultato ma può influire leggermente 
sui tempi di compilazione.

Tra le diverse opzioni di CMake si nota intanto \texttt{ANDROID\_ABI}, che specifica l'ABI per il quale compilare il
progetto. Un ABI (\textit{Application Binary Interface}) rappresenta sostanzialmente l'insieme di set di istruzioni
supportate da un dispositivo, come spiegato in \cite{adev_abi}. 
Si limita la scelta ad \texttt{armeabi-v7a} e \texttt{arm64-v8a}, rispettivamente le ABI delle architetture ARM a 32 e 64 bit. 
Architetture differenti da ARM non verrebbero comunque testate su Android.

Altre opzioni interessanti sono quelle riguardanti le ottimizzazioni hardware: ad esempio \texttt{WITH\_TBB}, 
\texttt{ENABLE\_NEON} o \texttt{WITH\_OPENCL}. Per abilitare le ottimizzazioni è sufficiente settare a \texttt{ON}
l'opzione di interesse.

\subsection{Installazione}

Una volta ottenuti i binari di tutti i moduli necessari di OpenCV (tramite cross-compilazione o download), è possibile
procedere con l'inserimento all'interno del progetto di Android Studio: come già detto, queste librerie dovranno essere 
parte integrante dell'applicazione.

L'inserimento è molto meccanico e avviene con la creazione di due cartelle nel progetto: una per contenere gli header di
OpenCV, l'altra per contenere le librerie dinamiche contenenti il codice (file \texttt{.so}).
Sia gli header che le librerie vengono inclusi nel progetto aggiungendo istruzioni ad hoc all'interno del file 
\texttt{CMakeLists.txt} generato automaticamente da Android Studio per compilare la componente nativa del progetto che
comunichi con gli entry point Java.
Segue un esempio con le principali istruzioni CMake utilizzate per aggiungere le librerie di OpenCV:

\usemintedstyle{vs}
\begin{minted}[
    frame=lines,
    framesep=2mm,
    baselinestretch=1.2,
    bgcolor=bg,
    fontsize=\footnotesize
]{cmake}

include_directories(${CMAKE_SOURCE_DIR}/../jniIncludes)

add_library( libopencv_core SHARED IMPORTED )
set_target_properties( libopencv_core PROPERTIES IMPORTED_LOCATION
        ${CMAKE_SOURCE_DIR}/../cmakeLibs/${ANDROID_ABI}/libopencv_core.so)

#...

target_link_libraries( 
        # Specifies the target library.
            native-lib

        # Links the target library to the log library
        # included in the NDK.
        ${log-lib}

        libopencv_core
        #...
)

#...
\end{minted}

\section{Sviluppo dell'applicazione Android}

Il deployment e l'installazione di applicazioni su dispositivi Android avviene mediante l'uso di file .APK.
I file APK sono sostanzialmente degli archivi compressi in formato .ZIP il cui contenuto è quindi visualizzabile con
un qualsiasi gestore di archivi. Contengono tutti i file necessari per una singola applicazione Android: librerie native,
classi Java compilate, ma anche risorse e immagini.

La creazione dei file APK a partire dai file sorgente e dal progetto iniziale spetta solitamente a sistemi di compilazione.
Per facilitare questa operazione, viene di norma utilizzato un ambiente di sviluppo e quindi un IDE specializzato.\\
Nel caso di Android l'IDE consigliato è Android Studio, che fornisce un bundle di installazione contenente tutti gli strumenti
e le toolchain necessarie, facilmente configurabili.

Il linguaggio principale supportato da Android è \textbf{Java}: ogni applicazione richiede anche solo una minima parte scritta 
in Java per poter utilizzare le API di sistema e gestire i cosiddetti \emph{Entry Point} dell'applicazione, ovvero i meccanismi 
tramite i quali l'utente può avviare ed interagire con l'applicazione, l'interfaccia utente è infatti manipolabile esclusivamente
con l'utilizzo di classi Java.

È comunque disponibile il supporto per codice nativo C/C++ grazie all'NDK (\emph{Native Development Kit}): un insieme di strumenti
per la compilazione e il debug integrabili con Android Studio. Lo strumento di configurazione di default per codice nativo è 
CMake, tra i più diffusi per progetti C e C++: un esempio di configurazione tramite CMake è quello giusto esposto nella
sezione di cui sopra, in cui si aggiungono le librerie OpenCV al progetto.

Per configurare infine il progetto nella sua interezza, compresi i dettagli di deployment, Android Studio richiede l'utilizzo 
del plugin \textbf{Gradle}, che può essere integrato con CMake (in modo molto automatizzato) qualora sia prevista la 
presenza di codice nativo.

\subsection{Architettura dell'applicazione}

Il codice Java e quello nativo, all'interno di un'applicazione Android, comunicano grazie al framework \textbf{JNI} 
(\textit{Java Native Interface}). Questo framework consente in modo particolare l'invocazione di funzioni scritte in codice
nativo da parte di classi Java. 

Nonostante il funzionamento di JNI sia piuttosto complesso (si veda la documentazione in \cite{jnispec}), l'utilizzo di esso 
in un progetto Android Studio avviene in modo piuttosto guidato e trasparente nei confronti dello sviluppatore: tutti i 
dettagli interni di collegamento e configurazione tra i vari componenti vengono gestiti in maniera automatizzata. 
Inoltre, come si vedrà, ai fini di questo progetto è stato sufficiente fare uso delle funzioni più basilari di JNI.

Dopo un'attenta analisi, si è deciso di strutturare l'applicazione come segue:
\begin{itemize}
    \item Entry point Java (\texttt{MainActivity.java});
    \item Classe Java che implementi un \emph{thread} secondario per l'invocazione del codice OTV (\texttt{OTVThread.java});
    \item File C++ che faccia da wrapper per le funzionalità di OTV, interfacciandosi con il codice Java (file 
    \texttt{native-lib.cpp});
    \item Codice di OTV, pressoché inalterato (cartella \texttt{otv/src/}).
\end{itemize}

Il motivo per cui si è scelto di introdurre un thread apposito per l'invocazione di OTV è legato alle modalità con cui la 
classe MainActivity gestisce l'interazione con l'utente.
Ogni \textit{Activity} ha un ciclo di vita --- descritto in \cite{adev_lifecycle} --- basato su funzioni di callback invocate dal sistema
Android a seguito di determinate azioni compiute dall'utente, ad esempio il metodo \texttt{onCreate}, invocato non appena
l'utente apre l'applicazione.

Poiché OTV presenta le caratteristiche di un demone (ha un ciclo di esecuzione che idealmente non termina mai), inserirlo 
all'interno di un metodo di callback significherebbe deviare dal normale ciclo di esecuzione dell'attività.
La scelta di un thread che esegua OTV in background è quasi obbligata in quanto è l'unico modo di garantire un'esecuzione
(quasi) disaccoppiata dagli eventi della UI ma soprattutto che avvenga in modo asincrono rispetto alla creazione degli elementi
dell'interfaccia grafica: se si invocasse OTV all'interno del metodo \texttt{onCreate}, l'applicazione risulterebbe bloccata 
sulla creazione dell'interfaccia.

Nel metodo \texttt{onCreate} della classe \texttt{MainActivity} risulterà quindi:

\usemintedstyle{vs}
\begin{minted}[
    baselinestretch=1.2,
    fontsize=\footnotesize
]{java}
protected void onCreate(Bundle savedInstanceState) {
    //...
    OTVThread otvThread = new OTVThread(updateUIHandler);
    otvThread.start();

    //casella di testo nella UI
    tv.setText("OTV was launched in background...");
}
\end{minted}

Nella classe \texttt{OTVThread}:

\usemintedstyle{vs}
\begin{minted}[
    baselinestretch=1.2,
    fontsize=\footnotesize
]{java}
public void run() {
    String otv_results = runOTV();
    //...
}

// metodo nativo invocabile tramite JNI
public native String runOTV();
\end{minted}

\vspace{1em}

Infine, nel wrapper C++ (\texttt{native-lib.cpp}):

\vspace{1em}

\usemintedstyle{vs}
\begin{minted}[
    baselinestretch=1.2,
    fontsize=\footnotesize
]{cpp}

extern "C" JNIEXPORT jstring JNICALL 
Java_gullp_androidotv_OTVThread_runOTV( JNIEnv* env, jobject) {

    //costruzione dei parametri richiesti
    char* argv[6];
    argv[0] = "OTV_AndroidInstance";
    argv[1] = PATH_INFO_VIDEO;
    argv[2] = PATH_FILE_PARAMS;
    argv[3] = PATH_LK_FILE;
    argv[4] = PATH_OUTPUT;
    argv[5] = PATH_MASK;

    LOGD("Running OTV...");

    //effettiva invocazione del codice OTV nativo
    return env->NewStringUTF(run(6, argv).c_str());
}

\end{minted}

\vspace{1em}

I file di configurazione richiesti da OTV (contenenti i parametri di calibrazione per l'algoritmo di Lucas Kanade)
sono stati inseriti nella cartella \texttt{/data/data/\{nome\_package\}} del dispositivo, che rappresenta la porzione di 
storage interno dedicata all'applicazione.