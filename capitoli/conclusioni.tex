\chapter*{Conclusioni}
\label{cap:conclusioni}
\addcontentsline{toc}{chapter}{\nameref{cap:conclusioni}}

Il porting di OTV su sistema Android ha prodotto risultati decisamente promettenti, evidenziando performance e consumi
energetici anche parzialmente migliori rispetto a quanto ottenuto su Raspberry Pi.

Per fornire una valutazione completa e confrontare sotto ogni aspetto le due soluzioni, è necessario condurre un'analisi
più approfondita sui consumi energetici, specie considerando il ciclo di esecuzione completo di OTV che prevede le fasi
aggiuntive di acquisizione, estrazione ed infine di idle.\\
In particolare, la fase di idle risulta particolarmente critica nei sistemi Android, poiché essi --- al contrario dei Raspberry
--- non dispongono di una modalità di stand-by ma devono fare affidamento alle varie modalità di sistema meno sospensive quali 
il blocco dello schermo.

Qualora si decidesse di approfondire il presente lavoro, il primo passo sarebbe quindi quello di 
completare l'applicazione inserendo componenti dedicate all'estrazione, acquisizione, e soprattutto un'implementazione ad hoc
della fase di idle.

L'utilizzo di dispositivi Android nell'ambito di applicazione discusso potrebbe --- se confermato da ulteriori analisi ---
portare a prestazioni migliori, a fronte di costi comunque piuttosto contenuti e una configurazione molto semplificata.