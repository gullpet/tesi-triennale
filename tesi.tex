\documentclass[12pt,a4paper,openright]{report}

\usepackage[italian]{babel}
\usepackage{newlfont}
\usepackage{parskip}
\usepackage[utf8]{inputenc}
\usepackage{fancyhdr}
\usepackage{hyperref}
\usepackage{geometry}
\usepackage{babelbib}

%serve solo per il frontespizio
\textwidth=460pt \oddsidemargin=0pt

% Direttive fancyhdr, si può fare riferimento al layout:
% {lhead}   {chead}   {rhead}
%        <Page content>
% {lfoot}   {cfoot}   {rfoot}

\pagestyle{fancy}%\addtolength{\headwidth}{20pt}
\renewcommand{\chaptermark}[1]{\markboth{\thechapter.\ #1}{}}
\renewcommand{\sectionmark}[1]{\markright{\thesection \ #1}{}}
\rhead[\fancyplain{}{\bfseries\leftmark}]{\fancyplain{}{\bfseries\thepage}}
\lhead{\leftmark}
\rhead{\thepage}
\cfoot{}

\linespread{1.2}

\begin{document}

%frontespizio
% Frontespizio reperito da: https://corsi.unibo.it/magistrale/ScienzeInternet/tesi-in-latex
% e liberamente modificato da Guglielmo Palaferri


%\documentclass[12pt,a4paper]{report}
%\usepackage[italian]{babel}
%\usepackage{newlfont}
%
%\begin{document}

%\textwidth=450pt\oddsidemargin=0pt
\begin{titlepage}
\begin{center}
{{\Large{\textsc{Alma Mater Studiorum $\cdot$ Universit\`a di
Bologna}}}} \rule[0.1cm]{15.8cm}{0.1mm}
\rule[0.5cm]{15.8cm}{0.6mm}
{\small{\bf SCUOLA DI INGEGNERIA E ARCHITETTURA\\
Corso di Laurea in Ingegneria Informatica }}
\end{center}
\vspace{35mm}
\begin{center}
{\LARGE{\bf Porting di un algoritmo per la stima del flusso ottico su smartphone Android}}\\
\end{center}
\vspace{50mm}
\par
\noindent
\begin{minipage}[t]{0.47\textwidth}
{\large{\bf Relatore:\\
Prof.\\
Stefano Mattoccia}}
\end{minipage}
\hfill
\begin{minipage}[t]{0.47\textwidth}\raggedleft
{\large{\bf Candidato:\\
Guglielmo Palaferri}}
\end{minipage}
\vspace{20mm}
\begin{center}
{\large{\bf Appello II\\%inserire il numero della sessione in cui ci si laurea
Anno Accademico 2020-2021}}%inserire l'anno accademico a cui si è iscritti
\end{center}
\end{titlepage}

%\end{document}

\newpage
\clearpage{\pagestyle{empty}\cleardoublepage}

\pagenumbering{roman} %numerazione romana per le pagine di prefazione

%il textwidth=455pt serve per il frontespizio, è troppo grande per il testo
\newgeometry{textwidth=426pt}

\shipout\null
\chapter*{Introduzione}
\addcontentsline{toc}{chapter}{Introduzione}
\lhead{INTRODUZIONE}

Il monitoraggio costante della velocità di fiumi e correnti d'acqua può assumere notevole importanza sia nello studio di 
fenomeni idrologici puramente naturali, sia nella progettazione di opere ingegneristiche strettamente legate ad un 
particolare flusso d'acqua. Ad esempio, può aiutare ad analizzare e rilevare fenomeni come le inondazioni (specie gli 
avvenimenti improvvisi, che destano particolare attenzione), così come anche il trasporto di sedimenti o 
l'erosione delle rocce.

Molte delle tecniche tradizionali utilizzate per l'osservazione di un flusso idrico, tuttavia, non garantiscono 
grande efficienza e presentano costi elevati: spesso è richiesta la presenza di personale specializzato per la 
manutenzione di dispositivi complessi.\cite{rs10122010}\\ %citazione a paper remote sensing
Una soluzione che preveda invece l'installazione di apparecchi ottici, e basi quindi il monitoraggio sull'elaborazione di
immagini, può consentire di abbattere notevolmente i costi e di distribuire il sistema di osservazione ottenendo quindi 
maggiore resistenza ai guasti.%inserire immagine di esempio applicazione OTV.

\textit{\underline{immagine esempio applicazione OTV}}

È proprio questo un caso di utilizzo di \textbf{OTV} (\textit{Optical Tracking Velocimetry}), una tecnica che fa uso di 
particolari algoritmi di computer vision (in particolare l'algoritmo di Lucas-Kanade, utilizzato per la stima del flusso 
ottico) per tracciare le traiettorie e le velocità del flusso d'acqua a partire da una serie di immagini. 
Il tracciamento viene svolto grazie al riconoscimento di particelle quali detriti e altri residui e al confronto di fotogrammi 
consecutivi.

Il metodo OTV è pensato per essere applicato a dispositivi di elaborazione a basso costo e di dimensioni contenute: questi
sarebbero posizionati lungo corsi d'acqua in aree geografiche remote. I dati poi raccolti da questi dispositivi potranno essere
spediti (tramite meccanismi semplici come l'invio di SMS) ad un sistema di raccolta dati centralizzato.
Va da sé dunque che l'ottimizzazione dei consumi energetici dei dispositivi costituisca un punto cruciale per la 
realizzabilità di un tale sistema di monitoraggio. Questo tema verrà preso in considerazione(?) e rappresenta uno dei punti
principali degli studi finora condotti sull'argomento.\\

L'algoritmo è stato inizialmente testato su dispositivi della famiglia \textit{Raspberry}, per via delle loro dimensioni molto
contenute e in generale per le funzionalità da essi offerte, molto coerenti con i requisiti del progetto. Le analisi hanno
riportato ottimi risultati dal punto di vista dei consumi energetici in particolare dei modelli Raspberry Pi 3B e 4.\cite{app11157027} 

Altri dispositivi con buone potenzialità e con un profilo che si presti bene al contesto di utilizzo sono gli \textbf{smartphone},
con particolare riferimento a quelli basati su sistema operativo \textbf{Android}. L'utilizzo di tali dispositivi richiede ovviamente
una seppur minima quantità di modifiche rispetto al deployment effettuato su Raspberry, ed è proprio questo il tema centrale
del seguente documento.\\
Nei prossimi capitoli si procede quindi --- dopo aver introdotto qualche informazione necessaria su OTV --- a descrivere 
la realizzazione di un'applicazione per smartphone Android che adatti l'implementazione in C++ di OTV (disponibile su GitHub)
e i risultati in termini di prestazioni (e consumi energetici?) che ne sono conseguiti.




\clearpage{\pagestyle{empty}\cleardoublepage}

\tableofcontents
%\rhead[\fancyplain{}{\bfseries\leftmark}]{\fancyplain{}{\bfseries\thepage}}
%\lhead[\fancyplain{}{\bfseries\thepage}]{\fancyplain{}{\bfseries
%INDICE}}
\clearpage{\pagestyle{empty}\cleardoublepage}

\pagenumbering{arabic} %numerazione araba per il contenuto effettivo

%++++++++++++

\lhead{\leftmark}
\rhead{\thepage}

\chapter{OTV}
\section{Contesto di utilizzo}

Come già brevemente descritto, OTV prevede un deployment su dispositivi di dimensioni ridotte e autosufficienti dal punto di
vista energetico. In particolare, la configurazione testata su Raspberry introduceva i seguenti componenti:
\begin{itemize}
    \item Raspberry Pi 3B/4 per l'elaborazione
    \item Panello solare 6 W (PiJuice Solar Panel) per sostenere i consumi energetici
    \item Batteria esterna (PiJuice Hat) per fornire alimentazione
\end{itemize}
\cite{app11157027} Una simile configurazione verrebbe usata con smartphone, salvo ovviamente l'utilizzo di una batteria 
aggiuntiva.

\underline{\emph{*Immagine applicazione OTV 2 (quella con lo sketch del fiume usata nel paper nuovo)*}}

\section{Ciclo di funzionamento}
Il dispositivo così composto, una volta accuratamente posizionato ed avviato, dovrebbe eseguire \emph{quattro} misurazioni 
della velocità dell'acqua ogni ora, risultando quindi a regime in un ciclo di funzionamento periodico della durata di 15 minuti.

Sebbene la misurazione mediante l'algoritmo OTV sia svolta sul momento, non viene effettuata sulle immagini direttamente ricevute
e lette in input dalla telecamera: il video acquisito necessita di una fase preliminare che prepari le immagini per essere elaborate.
Questo viene fatto, tra le altre cose, per consentire di scegliere un settaggio particolare (ad esempio, selezionare una 
risoluzione diversa rispetto al video originale), utile successivamente al fine di ottimizzare l'elaborazione.

Il ciclo di funzionamento si articola quindi in questo modo:
\begin{enumerate}
    \item Fase di \textbf{acquisizione}: le immagini vengono acquisite dalla telecamera. Questa fase ha una durata fissa e dipende
    dalla lunghezza del video che si vuole analizzare: tipicamente 20 secondi.
    \item Fase di \textbf{estrazione} dei frame: a partire dal video acquisito, si estraggono i fotogrammi che lo compongono a seconda
    della configurazione scelta, in particolare è possibile specificare la risoluzione desiderata tra:
    \begin{itemize}
        \item Full Resolution (\textbf{F}): Risoluzione originale
        \item Half Resolution (\textbf{H}): Risoluzione dimezzata
        \item Quarter Resolution (\textbf{Q}): Risoluzione 1/4 dell'originale
    \end{itemize}
    \item Fase di \textbf{elaborazione} (OTV): a questo punto le immagini estratte vengono effettivamente elaborate utilizzando
    OTV. Questa fase è cruciale dal punto di vista dei consumi in quanto è quella che può variare maggiormente a seconda della
    configurazione usata e delle ottimizzazioni implementate. È bene quindi analizzarla di conseguenza.
    \item Fase di \textbf{idle}: una volta conclusa l'elaborazione (ed eventualmente spediti i dati rilevati) segue un periodo di stand-by,
    in cui si attende il tempo necessario prima della prossima rilevazione. Anche questa fase è molto importante per determinare
    i consumi energetici del processo: se il dispositivo dovesse disporre di una modalità di risparmio energetico, 
    l'energia utilizzata potrebbe diminuire drasticamente.
\end{enumerate}

Le fasi su cui è possibile effettivamente lavorare per ottenere risultati migliori sono quelle di elaborazione (in modo particolare)
e di idle.

\section{Ottimizzazioni}

Le ottimizzazioni applicabili ad OTV analizzate in \cite{rs12122047} consistono in una serie di tecniche e meccanismi che
possano contribuire ad aumentare l'efficienza energetica del sistema. %e in generale ad abbassarne i consumi. 
Tra queste, possiamo distinguere quelle legate al \textbf{software} e quelle invece a livello \textbf{hardware} 
(ad esempio, l'utilizzo di istruzioni particolari).

Le ottimizzazioni software consistono essenzialmente nella configurazione del dispositivo in modo ad esempio
da disattivare le opzioni software che risultino superflue (Wi-Fi, Bluetooth ecc.). Si parla di ottimizzazioni derivanti dal
sistema operativo utilizzato e dunque dipendenti dal dispositivo in questione. Si vedranno ottimizzazioni di questo tipo
esclusive al sistema Android.

\subsection{Ottimizzazioni hardware}

Nel caso delle ottimizzazioni hardware, si parla di particolari metodi che introducono differenti modelli di esecuzione
a livello di processore, facendo leva specialmente sulla parallelizzazione delle istruzioni. Questo, oltre agli ovvi vantaggi
in termini di performance, può portare ad una maggiore efficienza in termini di consumi.
Si delineano tre possibilità principali, eventualmente sovrapponibili, focalizzate su aspetti e modalità diverse di 
parallelizzazione:
\begin{itemize}
    \item Esecuzione multi-core mediante \textbf{OpenMP} o \textbf{TBB}: viene sfruttata l'architettura a più core per
    ottenere il più tipico livello di parallelismo. 
    \item Esecuzione di istruzioni \textbf{SIMD} tramite \textbf{NEON}
    \item Esecuzione su \textbf{GPU} mediante la libreria \textbf{OpenCL}
\end{itemize}

\subsubsection{OpenMP e TBB}




\clearpage{\pagestyle{empty}\cleardoublepage}

%===========

%\chapter{Processo di sviluppo}
%\section{Prima sezione}




%\clearpage{\pagestyle{empty}\cleardoublepage}

%===========

%\chapter{Testing e misurazioni}
%\section{Prima sezione}




%\clearpage{\pagestyle{empty}\cleardoublepage}

%===========

%\chapter*{Conclusioni}




%\clearpage{\pagestyle{empty}\cleardoublepage}


% --------------- Fine contenuto, inizio bibliografia ---------------

\bibliographystyle{abbrv}
\bibliography{refs}

%\begin{thebibliography}{90}             %crea l'ambiente bibliografia

    %\rhead[\fancyplain{}{\bfseries \leftmark}]{\fancyplain{}{\bfseries
    %\thepage}}
    %\addcontentsline{toc}{chapter}{Bibliografia}

    %\bibitem{K1} .
    %\bibitem{K2} Secondo oggetto bibliografia.
    %\bibitem{K3} Terzo oggetto bibliografia.
    %\bibitem{K4} Quarto oggetto bibliografia.

%\end{thebibliography}

\end{document}